\chapter{An extract from an essay using the Harvard referencing system}
\begin{refsection}
\textelp{} The literal adaptation of a book to film is practically impossible. As \textcite[4]{Stam2005} suggests:
\blockquote{The shift from a single-track verbal medium such as the novel to a multi-track medium like film, which can play not only with words (written and spoken) but also with music, sound effects, and moving photographic images, explains the unlikelihood and \textelp{} undesirability of literal fidelity.}
It is puzzling, then, that readers and audiences are so critical of adaptations which take liberties, sometimes for the better, with their source material.

Film adaptations of novels are frequently 
\blockquote[{\parencite[15]{Stam2005a}}]{castigated and held to an absurdly rigorous standard of fidelity}. 
If key scenes from a novel are pruned for film, audiences often react negatively. However, fidelity is not an appropriate measure for evaluating a film adaptation’s success, as numerous scholars concur \parencite{Desmond2006,Leitch2008,McFarlane1996,Miller2004}. Judging film adaptations is ultimately, \textcite[9]{Whelehan1999} contends, ‘an inexact science dogged by value judgments about the relative artistic worth of literature and film’. A fan of a novel might denigrate a film adaptation which alters the original book in some fashion, but their response is highly subjective and fails to take into account the practices and realities of film production \parencite[26]{McFarlane2007}.

Sometimes there are grounds for hostility. Author Alan Moore has witnessed a number of his complex graphic novels adapted into shallow Hollywood products, making him extremely critical of filmmakers and the filmmaking process \parencite{Ahsurst2009}. However, this kind of attitude can be knee-jerk and reactionary. Rather than being overly pedantic about textual faithfulness, it is best to approach film adaptations as re-interpretations of their source material \parencite[8]{Hutcheon2006} or as 
% \textquote[(\citeauthor{Kristeva1995} cited in \cite[2]{Sanders2006})]{a permutation of text, an intertextuality}. % needs to make pag enumber work eg (Kristeva, cited in Sanders 2006, p. 2)
\blockquote[{\parencite[Kristeva, cited in][2]{Sanders2006}}]{a permutation of text, an intertextuality}. % needs to make pag enumber work eg (Kristeva, cited in Sanders 2006, p. 2)
Moreover, new modes of production further complicate existing definitions of, and approaches to, adaptation \parencite[180]{Moore2010}. So\textelp{}
\pagebreak[3]
\printbibliography[heading=subbibliography]
\end{refsection}